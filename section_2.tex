	\section*{نمونه گیری با طبقه بندی}
\subsection*{تعریف}
در نمونه گیری طبقه بندی جامعه می بایست به طبقات همگن طبقه بندی شود، به گونه ای که بین طبقات اختلاف وجود داشته باشد.
\subsection*{روش نمونه گیری}
به گونه ای است که از هر طبقه نمونه ای به روش 
\lr{SRS}
انتخاب می شود و 
انتخاب نمونه از هر طبقه
\underline{ مستقل}
از انتخاب نمونه از طبقه دیگر است.
\\
جامعه:
\begin{equation*}
	\begin{pmatrix}
		y_{1\,1} \\
		\vdots \\
		y_{1\,N_1} \\
		\rule{0.7cm}{0.2pt} \\
		y_{2\,1} \\
		\vdots \\
		y_{2\,N_2} \\
		\rule{0.7cm}{0.2pt} \\
		\vdots \\
		\rule{0.7cm}{0.2pt} \\
		y_{H\,1} \\
		\vdots \\
		y_{H\,N_H} \\
	\end{pmatrix}
	\longrightarrow 
	\begin{cases}
		N_1:y_{1\, 1},\dots , y_{1\, N_1}:\bar{y}_{1.} \\
		N_2:y_{2\, 1},\dots , y_{2\, N_2}:\bar{y}_{2.} \\
		\vdots \\
		N_H:y_{H\, 1},\dots , y_{H\, N_H}:\bar{y}_{H.} \\
	\end{cases}
\end{equation*}
نمونه:
\begin{equation*}
	\begin{pmatrix}
		y_{1\,1} \\
		\vdots \\
		y_{1\,n_1} \\
		\rule{0.7cm}{0.2pt} \\
		y_{2\,1} \\
		\vdots \\
		y_{2\,n_2} \\
		\rule{0.7cm}{0.2pt} \\
		\vdots \\
		\rule{0.7cm}{0.2pt} \\
		y_{H\,1} \\
		\vdots \\
		y_{H\,n_H} \\
	\end{pmatrix}
	\longrightarrow 
	\begin{cases}
		n_1:y_{1\, 1},\dots , y_{1\, n_1}:\bar{y_{1}} \\
		n_2:y_{2\, 1},\dots , y_{2\, n_2}:\bar{y_{2}} \\
		\vdots \\
		n_H:y_{H\, 1},\dots , y_{H\, n_H}:\bar{y}_{H} \\
	\end{cases}
\end{equation*}
\subsection*{چند تعریف:}
\begin{enumerate}
	\item میانگین جامعه:
	\begin{equation*}
		\bar{y}_N\footnotemark[1]=\frac{\sum_{i=1}^{H} \sum_{j=1}^{N_i}y_{ij}}{N}
	\end{equation*}
	\item انحراف از معیار جامعه:
	\begin{equation*}
		S^2=\frac{\sum_{i=1}^{H} \sum_{j=1}^{N_i} (y_{ij}-\bar{y}_N)^2}{N-1}
	\end{equation*}
	\item میاگین طبقه 
	$i$
	ام:
	\begin{equation*}
		\bar{y}_{i.}=\frac{\sum_{j=1}^{N_i}y_{ij}}{N_i}
	\end{equation*}
	\item انحراف از معیار طبقه 
	$i$
	ام:
	\begin{equation*}
		S_i^2=\frac{\sum_{j=1}^{N_i}(y_{ij}-\bar{y}_{i.})^2}{N_i-1}
	\end{equation*}
	\footnotetext[1]{$N=\sum_{i=1}^{H}N_i$}
	\item میانگین نمونه ی طبقه 
	$i$
	ام:
	\begin{equation*}
		\bar{y}_{i}=\frac{\sum_{j=1}^{n_i}y_{ij}}{n_i}
	\end{equation*}
	\item انجراف از معیار نمونه 
	$i$
	ام:
	\begin{equation*}
		s_i^2=\frac{\sum_{j=1}^{n_i}(y_{ij}-\bar{y}_{i})^2}{n_i-1}
	\end{equation*}
\end{enumerate}
اگر قرار دهیم که 
$W_i=\frac{N_i}{N}$
آنگاه می توان گفت که 
$\bar{y}_{str}$
یک برآوردگر خوب برای 
$\bar{y_N}$
است.
\begin{equation*}
	\bar{y}_{str}=\sum_{i=1}^{H}W_i\, \bar{y_i}
\end{equation*}
ثابت کنید که:
\begin{equation*}
	E(\bar{y}_{str})=\bar{y}_N
\end{equation*}
برهان:
\begin{multline*}
	E(\bar{y}_{str})=E[\sum_{i=1}^{H}W_i\, \bar{y}_i]
	=\sum_{i=1}^{H}W_i\,E(\bar{y}_i)=
	\sum_{i=1}^{H}\frac{N_i}{N} \bar{y}_{i.}=\frac{1}{N}\sum_{i=1}^{H}N_i\,\bar{y}_{i.}
	=\\=\frac{\sum_{i=1}^{H}\cancel{N_i}\frac{1}{\cancel{N_i}}\sum_{j=1}^{N_i}y_{ij}}{N}
	=\frac{\sum_{i=1}^{H}\sum_{j=1}^{N_i}y_{ij}}{N}=\bar{y}_N
\end{multline*}
مقدار خطای برآورد را پیدا کنید. 
\\
از آنجایی که برآوردگر ما اریب است، مقدار خطای آن با واریانس آن برابر است:
\begin{multline*}
	Var(\bar{y}_{str})
	=Var[\sum_{i=1}^{H}W_i\,\bar{y}_i]
	=\sum_{i=1}^{H}W_i^2\, Var{\bar{y}_i}+\underset{indpendent\Rightarrow 0}{\underbrace{\underset{i\neq j}{\sum^H\sum^H}\, Cov(\bar{y}_i,\bar{y}_j)}}
	=\\=
	\sum_{i=1}^{H}W_i^2\, (1-\frac{n_i}{N_i})\frac{s_i^2}{n_i}\footnotemark[2]
	=\sum_{i=1}^{H}\frac{N_i^2}{N^2}(1-\frac{n_i}{N_i})\frac{s_i^2}{n_i}
	=\frac{1}{N^2}\sum_{i=1}^{H}N_i(N_i-n_i)\frac{s_i^2}{n_i}\footnotemark[4]
	=\\=
	\sum_{i=1}^{H}W_i^2\, \frac{S_i^2}{n_i}-\frac{1}{N}\sum_{i=1}^{H}W_i\, S_i^{2}\,\footnotemark[3]
\end{multline*}
\footnotetext[2]{فرمول 1}
\footnotetext[3]{فرمول 2}
\footnotetext[4]{فرمول 3}
\section*{تعیین حجم نمونه }
\subsection*{نمونه گیری با تخصیص متناسب}
\begin{gather*}
	n_i\varpropto N_i \\
	n_i=kN_i \\
	\sum_{i=1}^{H}n_i=k\sum_{i=1}^{H}N_i \\
	n=kN \Rightarrow k=\frac{n}{N}
\end{gather*}
اگر در روش نمونه گیری طبقه بندی، حجم نمونه ای که از طبقات می گیریم متناسب با حجم طبقات باشد، داریم:
\begin{align*}
	n_i&=n.W_i\footnotemark & \frac{n_i}{N_i}&=\frac{N_i}{N}
\end{align*}
\footnotetext{$W_i=\frac{N_i}{N}$}
به این روش نمونه گیری طبق بندی با تخصیص متناسب می گویند.
\begin{multline*}
	Var(\bar{y}_str)=\sum_{i=1}^{H}W_i^2(1-\frac{n_i}{N_i})\frac{S_i^2}{n_i}=
	\sum_{i=1}^{H}W_i.\frac{N_i}{N}(1-\frac{n_i}{N_i})\frac{S_i^2}{n_i}
	=\\ \overset{\frac{N_i}{N}=\frac{n_i}{n}}{=}
	\sum_{i=1}^{H}W_i.\frac{\cancel{n_i}}{n}(1-\frac{n_i}{N_i})\frac{S_i^2}{\cancel{n_i}}
	=(1-\frac{n}{N})\frac{\sum_{i=1}^{H}W_iS_i^2}{n}
\end{multline*}
فرض کنید که 
$S_i^2=S_0^2$
:
\begin{gather*}
	(1-\frac{n}{N})\frac{\sum_{i=1}^{H}W_iS_i^2}{n}=(1-\frac{n}{N})\frac{\sum_{i=1}^{H}W_iS_0^2}{n}
	=(1-\frac{n}{N})\frac{S_0^2\sum_{i=1}^{H}W_i}{n}
	=\\=
	(1-\frac{n}{N})\frac{S_0^2}{n}
\end{gather*}
در حقیقت با فرض 
$S_i^2=S_0^2$
،طبقه ها را یکسان فرض کردیم و نمونه گیری ما همان 
\lr{SRS}
بدون جایگزاری است.
\subsection*{تمرین:}
فرض کنید منطقه ای دارای 64 شهر است. منطقه را به دو طبقه تقسیم کرده ایم. طبقه اول 16 شهر و طبقه ی دوم 48 شهردارد.
هدف برآورد جمعیت منطقه بر اساس نمونه ای به حجم 24 شهر است. به منظور از سه روش زیر استفاده شده است.
\begin{enumerate}
	\item \lr{SRS}
	\item طبقه بندی با تخصیص متناسب
	\item طبقه بندی با انتخاب 12 شهر از هر منطقه
	دقت سه روش را محاسبه کنید.
	\\
	\begin{LTR}
		\begin{tabular}{c|cc}
			$i$ & $\sum_{j=1}^{N_i}y_{ij}$ & $\sum_{j=1}^{N_i}y_{ij}^2$ \\
			\hline
			$1$ & $10070$ & $7145450$ \\
			$2$ & $9408$ & $2141720$
		\end{tabular}
	\end{LTR}
	\begin{align*}
		T_N&=19568 & S_1^2&=53843 \\
		S^2&=52445 & S_2^2&=5581
	\end{align*}
	\begin{enumerate}
		\item روش اول
		\\
		\begin{align*}
			n_1&=\frac{24}{64}\times 16=6 & n_2&=\frac{24}{64}\times 48=18
		\end{align*}
		\begin{multline*}
			Var(\hat{T}_N)=Var(N\bar{y}_str)
			=\\=
			N^2Var(\bar{y}_str)=N^2(1-\frac{n}{N})\frac{S^2}{n}
			=(64)^2\times (1-\frac{24}{64})\frac{52445}{24}=(2365)^2
		\end{multline*}
		\item  روش دوم:
		\\
		\begin{multline*}
			Var(\hat{T}_N)=N^2(1-\frac{n}{N})\frac{\sum_{i=1}^{2}W_iS_i^2}{n}
			=(64)^2(1-\frac{24}{64})\frac{\frac{1}{4}(53843)+\frac{3}{4}(5581)}{24}
			=(1372)^2
		\end{multline*}
		\item روش سوم:
		\begin{multline*}
			Var(\hat{T}_N)=N^2Var(\bar{y}_str)=
			N^2\sum_{i=1}^{2}W_i^2(1-\frac{n_i}{N_i})\frac{S_i^2}{n_i}
			=\\=
			(64)^2\{(\frac{16}{64})^2(1-\frac{12}{16})\frac{53843}{12}+(\frac{48}{64})^2(1-\frac{12}{48})\frac{52445}{12}\}=(1024)^2
		\end{multline*}
	\end{enumerate}
	دقت ها:
	\\
	روش سوم
	$<$
	روش دوم
	$<$
	روش اول
	\subsection*{طبقه بندی با تخصیص بهینه}
	تابع هزینه:
	\begin{equation*}
		C=C_0+\sum_{i=1}^{H}C_in_i
	\end{equation*}
	\begin{itemize}
		\item[$\bullet$] $C$ :
		هزینه کل
		\item[$\bullet$] $C_0$ :
		هزینه ی ثابت
		\item[$\bullet$] $C_i$ :
		هزینه برای هر نمونه از طبقه 
		$i$
		ام
	\end{itemize}
	می خواهیم 
	$n_i$
	را به گونه ای تعیین کنیم، که با هزینه ی ثابت
	$C$
	، واریانس برآورد کمینه شود. 
	با استغاده از روش لاگرانژ داریم که:
	\begin{gather*}
		L=\sum_{i=1}^{H}W_i^2 \frac{S_i^2}{n_i}-\sum_{i=1}^{H}W_i\frac{S_i^2}{N}+\lambda(C_0+\sum_{i=1}^{H}C_in_i-C)
		\\
		\frac{\partial L}{\partial n_1}=-W_1^2\frac{S_1^2}{n_1^2}+\lambda C_1=0 
		\Rightarrow n_1=\frac{W_1S_1}{\sqrt{\lambda C_1}}
		\\
		\frac{\partial L}{\partial n_2}=-W_2^2\frac{S_2^2}{n_2^2}+\lambda C_2=0 
		\Rightarrow n_2=\frac{W_2S_2}{\sqrt{\lambda C_2}}
		\\
		\vdots \\
		\frac{\partial L}{\partial n_H}=-W_H^2\frac{S_H^2}{n_H^2}+\lambda C_H=0 
		\Rightarrow n_H=\frac{W_HS_H}{\sqrt{\lambda C_H}}
		\\
		\frac{\partial L}{\partial \lambda}=C_0+\sum_{i=1}^{H}C_in_i-C=0
		\end{gather*}
		پیدا کردن 
		$\lambda$
		:
		\begin{gather*}
			\sum_{i=1}^{H}n_i=\sum_{i=1}^{H}\frac{W_iS_i} {\sqrt{\lambda} \sqrt{C_i} } \\
			n=\frac{1}{\sqrt{\lambda}\sum_{i=1}^{H}\frac{W_iS_i}{\sqrt{C_i}}  } \\
			\sqrt{\lambda}=\frac{1}{n}\sum_{i=1}^{H}\frac{W_iS_i}{\sqrt{C_i}} \\
			\Rightarrow n_i=\frac{W_iS_i}{\frac{1}{n}\sum_{i=1}^{H}\frac{W_iS_i}{\sqrt{C_i}}\sqrt{C_i}}
			=\frac{\frac{W_iS_i}{\sqrt{C_i}}}{\sum_{i=1}^{H}\frac{W_iS_i}{\sqrt{C_i}}}\times n
		\end{gather*}
		مقدار کمینه واریانس بر اساس روش طبقه بندی با تخصیص بهینه:
		\begin{multline*}
			Var(\bar{y}_str)=
			\sum_{i=1}^{H}W_i^2\frac{S_i^2}{n_i}-\frac{\sum_{i=1}^{H}W_i^2S_i^2}{N}
			=\sum_{i=1}^{N}\frac{W_i^{\cancel{2}}S_i^{\cancel{2}}}
			{\frac{\frac{\cancel{W_i}\cancel{S_i}}{\sqrt{C_i}}}{\sum_{i=1}^{H}\frac{W_iS_i}{\sqrt{C_i}}}\times n}
			-\frac{\sum_{i=1}^{H}W_iS_i^2}{N}
			=\\=
			\frac{1}{n}(\sum_{i=1}^{H}\frac{W_iS_i}{\sqrt{C_i}})\sum_{i=1}^{H}W_i S_i \sqrt{C_i}
		\end{multline*}
		پیدا کردن مقدار 
		$n$
		مناسب:
		\begin{multline*}
			C=C_0+\sum_{i=1}^{H}\frac{\frac{W_iS_i}{\sqrt{C_i}}}{\sum_{i=1}^{H}\frac{W_iS_i}{\sqrt{C_i}}}\times n\times C_i
			\Rightarrow C-C_0=n\times \frac{1}{\sum_{i=1}^{H}\frac{W_iS_i}{\sqrt{C-i}}}\sum_{i=1}^{H}W_iS_i\sqrt{C_i}
			=\\=
			n\times\frac{\sum_{i=1}^{H}W_iS_i\sqrt{C_i}}{\sum_{i=1}^{H}\frac{W_iS_i}{\sqrt{C-i}}}
			\Rightarrow n=(C-C_0)\frac{\sum_{i=1}^{H}\frac{W_iS_i}{\sqrt{C_i}}}{\sum_{i=1}^{H}W_iS_i\sqrt{C_i}}
		\end{multline*}
\end{enumerate}
